\chapter*{Introduction}
\addcontentsline{toc}{chapter}{Introduction}
\markright{INTRODUCTION}

Very few results in mathematics are as thought-provoking as
Gödel's Incompleteness Theorems. Their discovery in the twentieth
century effectively cut short Hilbert's program of formalizing
all of mathematics into a complete and decidable axiomatic
system, whose most important achievements were Russell and
Whitehead's \textit{Principia Mathematica} and Zermelo--Fraenkel
set theory \cite{Zach2021}.

In very few words, Gödel's Incompleteness Theorems state,
respectively:
\begin{enumerate}
    \item No axiomatizable, consistent theory that is
        \enquote{strong enough}\footnote{This ends up amounting
            to it being able to represent all computable
            functions and relations, which, it turns out, is
        achieved by most interesting theories.} is complete.
        That is, there is some sentence $A$ in its language for
        which it cannot prove $A$ nor $\lnot A$.
    \item Consistent axiomatizable theories that are
        \enquote{strong enough} do not prove their own
        consistency statements.
\end{enumerate}

