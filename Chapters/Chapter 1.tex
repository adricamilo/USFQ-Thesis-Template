\chapter{Turing machines}

\epigraph{\itshape\myopeningquote Although axiomatizability and
    completeness implies decidability, incompleteness does not
    imply undecidability.\myclosingquote}
{Richard Zach \cite{Zach2021}}

\section{Machinery}

\subsection{Formal definition}

\begin{defn}
    A \textbf{Turing machine} $Z$ consists of:
    \begin{enumerate}
        \item A finite set of symbols $\qty{S_0, S_1,\dotsc,
            S_m}$ called the \textit{alphabet of $Z$}, where
            $S_0$ is a special symbol we will call
            \textit{blank}.
        \item  Some other stuff.
    \end{enumerate}
\end{defn}

\begin{defn} \label{def:basic_move}
    If $Z$ is a Turing machine, and $\alpha$ and $\beta$ are two
    instantaneous descriptions of $Z$, then we write the
    \textbf{basic move}
    \begin{equation*}
        \alpha \to \beta\quad (Z),
    \end{equation*}
    or simply $\alpha \to \beta$ when there is no ambiguity, to
    mean...
\end{defn}

\subsection{Some consequences}

From our definition of a basic move, and our requirement that all
the quadruples of a Turing machine have different combinations of
the first two symbols, we immediately get 

\begin{theorem} \label{thm:unique_move}
    Let $Z$ be a Turing machine. For any instantaneous
    description $\alpha$ of $Z$, there is at most one
    instantaneous description $\beta$ of $Z$ such that there is a
    basic move $\alpha \to \beta$.
\end{theorem}

\begin{theorem}
    If
    \begin{equation*}
        \alpha \to \beta\quad (Z) \qand \alpha \to \gamma\quad
        (Z),
    \end{equation*}
    then $\beta = \gamma$.
\end{theorem}

\subsection{More stuff}

\begin{theorem}
    If $M$ is a finite sequence of symbols, and $\mathcal{M}$ is
    a finite sequence of finite sequences of symbols, then their
    Gödel numbers are different.
\end{theorem}

\begin{proof}
    If $M$ is a finite sequence of symbols, then we can write
    \begin{equation*}
        \gn(M)=2^n m,
    \end{equation*}
    where $n$ and $m$ are odd integers. If $\mathcal{M}$ is a
    finite sequence of finite sequences of symbols, then its
    Gödel number has the form
    \begin{equation*}
        2^{n'}m',
    \end{equation*}
    where $n'$ is an even integer (because it itself is the Gödel
    number of a finite sequence of symbols), and $m'$ is an odd
    integer. By the Fundamental Theorem of Arithmetic, the Gödel
    numbers of $M$ and $\mathcal{M}$ are different.
\end{proof}

In the next chapter, we will see some other stuff.

